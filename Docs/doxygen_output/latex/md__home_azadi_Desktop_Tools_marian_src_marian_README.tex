\href{https://gitter.im/amunmt/marian?utm_source=badge&utm_medium=badge&utm_campaign=pr-badge&utm_content=badge}{\tt } \href{http://vali.inf.ed.ac.uk/jenkins/job/marian-train/}{\tt }

Google group for commit messages\+: \href{https://groups.google.com/forum/#!forum/mariannmt}{\tt https\+://groups.\+google.\+com/forum/\#!forum/mariannmt}

{\itshape marian-\/train} is a C++ G\+P\+U-\/specific parallel automatic differentiation library with operator overloading. It is the training framework used in the Marian toolkit.

Named in honour of Marian Rejewski, a Polish mathematician and cryptologist.

\subsection*{Website}

More information on \href{https://amunmt.github.io}{\tt https\+://amunmt.\+github.\+io}

\subsection*{Acknowledgements}

The development of Marian received funding from the European Union\textquotesingle{}s {\itshape Horizon 2020 Research and Innovation Programme} under grant agreements 688139 (\href{http://www.summa-project.eu}{\tt S\+U\+M\+MA}; 2016-\/2019), 645487 (\href{http://www.modernmt.eu}{\tt Modern MT}; 2015-\/2017) and 644333 (\href{http://tramooc.eu/}{\tt Tra\+M\+O\+OC}; 2015-\/2017), the Amazon Academic Research Awards program, and the World Intellectual Property Organization.

This software contains source code provided by N\+V\+I\+D\+IA Corporation. 